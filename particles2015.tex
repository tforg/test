\documentclass{particles2015}

%\AtBeginDocument{\vspace{3 cm}}
\usepackage{graphicx}
\usepackage{amsmath}
\usepackage{amsfonts}
\usepackage{natbib}
\usepackage{float}
\usepackage{booktabs} %horizontale Linien in Tabellen 
\usepackage{array} %notwendig für neue column-Stile
\usepackage{graphicx} 
\usepackage{caption}  
\usepackage{footnote} % footnote cross reference
\usepackage{amsmath}
\usepackage{psfrag}
\usepackage{subfig}
\usepackage{textcomp}

\usepackage{setspace}
\bibliographystyle{unsrt}

\usepackage{amssymb}

%\bibliography{Literature_20_5_15}
%\usepackage{biblatex}
\AtBeginDocument{\vspace{5 cm}}
\title{\textit{ParScale} - An Open-Source Library for the Simulation of Intra-Particle Heat and Mass Transport Processes in Coupled Simulations}

\author{Stefan Radl$^{1}$, Thomas Forgber$^{1}$, Andreas Aigner$^{2}$ and Christoph Kloss$^{2}$}

\heading{Stefan Radl, Thomas Forgber, Andreas Aigner and Christoph Kloss}

\address{
$^{1}$ Institute of Process and Particle Engineering\\
Graz University of Technology, Graz, Austria\\
Email: \underline{radl@tugraz.at}, \underline{thomas.forgber@tugraz.at}\\
Web page: http://ippt.tugraz.at
\and
$^{2}$ DCS Computing GmbH\\
Altenbergerstr. 66a – Science Park\\
4040 Linz, Austria\\
Email: \underline{andreas.aigner@dcs-computing.com}, \underline{christoph.kloss@dcs-computing.com}\\
Web page: http://www.dcs-computing.com\\}


\keywords{Granular Materials, DEM, \AtBeginDocument{\texttt{LIGGGHTS}\textregistered}, ParScale, Resolved Intra Particle Profiles, Heat and Mass Transfer, Sheared Bed}

\abstract{\textit{In this paper we provide an overview about the newly developed open-source library \texttt{ParScale}. We present the basic underlying equations the coupling capability to the existing DEM solver \AtBeginDocument{\texttt{LIGGGHTS}\textregistered}. Based on the coupling we perform simulations of sheared granular material over a wide range of Biot numbers and particle volume fractions. In addition certain verification cases will proof the functionality of \texttt{ParScale} as a stand-alone tool.}}

\AtBeginDocument{\vspace{5 cm}}

\begin{document}

\maketitle
\thispagestyle{empty}

\section{Introduction}
\label{sec:introduction}
Over the last ten years the coupling of Discrete Element Method (DEM) and Computational Fluid Dynamics (CFD) became state of the art \cite{Kloss2012}. These tools are able to resolve phenomena on a particle level with high computational efficiency and also provide a detailed understanding of the general flow behavior \cite{Wu2014}. However, these tools do not take intra particle properties profiles into account. In e.g. reative systems or complex reaction-diffusion problems, these spatially resolved properties are of key importance \cite{Dixon2007}. With \texttt{ParScale}, a novel developed library in a C/C++ environment and publicly available through \textit{www.github.com} \cite{ParScale_Public}, it is possible to close this gap. In the current development state, \texttt{ParScale} holds various models regarding heat transfer, mass transfer and grain-scales. Also, it is possible to resolve single and multi-reaction networks. Due to a modular class based structure and integration into an automated test harness, easy extendibility and software quality is ensured. Section \ref{sec:simu_method} outlines the basic underlying equations for intra particle phenomena which are used for the presented verification cases and the simple shear flow. In section \ref{sec:verification_case} the verification cases are presented in order to demonstrate and verification the functionality of \texttt{ParScale}. Section \ref{sec:simple_shear} demonstrates the coupling to \texttt{LIGGGHTS} $\textsuperscript{\textregistered}$ and evaluates the need of intra particle temperature profiles for a wide range of Biot numbers and particle volume fractions in a sheared bed application. 

\section{Simulation method and parallel coupling strategy}
\label{sec:simu_method}
One of the basic scopes of \texttt{ParScale} is to calculate the target property field as a function of time and space in a spherical particle for a fixed number of grid points. The basic underlying equations for transport phenomena within a particle are outlined in the next section.

\subsection{Transport within a particle}
For the transport within a single particle, the Fourier equation in spherical coordinates with $\lambda_{eff} = const.$ is used and shown in the following:
\begin{align}
\rho \, c_p \, \dfrac{\partial T}{\partial t} = div (\lambda_{eff}\, grad\,T) + \dot{q}^{\ast}
\label{eqn:heat_conduct_overall}
\end{align}
where $\rho$ is the density, $c_p$ is the heat capacity, $T$ represents the target property profile, $t$ is the time, $\lambda_{eff}$ is the effective thermal conductivity and $\dot{q}^{\ast}$ is a possible property source. By introducing the thermal diffusivity $a$ 
\begin{align}
a = \dfrac{\lambda_{eff}}{\rho \, c_p}
\end{align}
and constant thermal conductivity $\lambda_{eff}$ Eqn. (\ref{eqn:heat_conduct_overall}) results in
\begin{align}
 \dfrac{\partial T}{\partial t} =  a \cdot \left( \dfrac{\partial^2 T}{\partial r^2} + \dfrac{2}{r} \, \dfrac{\partial T}{\partial r} \right) + \dfrac{\dot{q}^{\ast}}{\rho \, c_p}
\label{eqn:heatconductradial}
\end{align}
where $r$ is the radius increment of the particle.\\
The resulting equation system are solved by a multi step, leading line based solver called \texttt{CVODE} which makes \texttt{ParScale} flexible and the algorithms stable in cases of fast changing environment conditions or stiff systems such as reactions \cite{Cohen1996}. For the second verification (see Section \ref{sec:heterogeneuous_reaction}) a simple heterogeneous reaction is evaluated and the basic governing equations are presented in the next section.

\subsection{Chemistry model}
Note that the following equations and notations are based on \cite{Noorman2011}. In the current developing state of \texttt{ParScale} all chemical analysis are made by the assumption of a generic reaction in which a solid species A and a gas species B yields a solid and gas species through the irreversible reaction:
\begin{align}
a A (s) + bB (g) \rightarrow cC(s) + sD(g)
\end{align}
Examples can be found e.g. at the reduction of iron oxides by $\text{H}_2$, calculations of ZnS and FeS with $\text{O}_2$ to metal oxides (ZnO, $\text{Fe}_2\text{O}_3$) and combustion of coal in order to give Co and ash. The mass balance equation for a single species in a porous particle is written as:
\begin{align}
\dfrac{\partial \varepsilon c_i}{\partial t} = - \dfrac{1}{r^2} \dfrac{\partial}{\partial r } (r^2 N_i) + s_i \text{ with } i = B, D
\end{align}
which could be solved together with the total mass equation
\begin{align}
\dfrac{\partial \varepsilon c}{\partial t} = - \dfrac{1}{r^2} \dfrac{\partial}{\partial r } (r^2 N) + \sum\limits_{j=1}^{NG} s_i
\label{eqn:total_concentration}
\end{align}
where $\varepsilon$ is the phase fraction, $c_i$ the concentration, $N_i$ the flux, $NG$ the number of gas components and $s_i$ a possible chemical source term. A typical model for $s_i$ is
\begin{align}
s_i = \sum\limits_{j=1}^{NR} \nu_{ij} \cdot \prod\limits_k^{NG+NS} c_k^{n_{k,j}}
\end{align} 
where $NR$ is the number of reaction, $\nu_{ij}$ is the stoichiometric coefficient, $NS$ is the number of solids and $n_{k,j}$ the reaction exponent. Note that in \texttt{ParScale} no total mass equation is solved at the moment. The mass balance equations for the solid phase are as follows:
\begin{align}
\dfrac{\partial c_i}{\partial t} = s_i \text{ with } i = A, C
\end{align}


For discretizing in physical space a second order central differencing scheme is used. For further information about the underlying equation, available models and discretizion schemes please refer to the online documentation in the public repository of \texttt{ParScale} \cite{ParScale_Public}. The next section briefly outlines the run modes of \texttt{ParScale} including the coupling to the open-source DEM solver \texttt{LIGGGHTS}\textsuperscript{\textregistered}.

\subsection{Parallel coupling strategy}
Besides a stand-alone mode the current developing state of \texttt{ParScale} provides coupling capabilities to \texttt{LIGGGHTS}\textsuperscript{\textregistered}  and selected solvers of \texttt{CFDEMcoupling}. In this paper we will only focus on the coupling to \texttt{LIGGGHTS}\textsuperscript{\textregistered}. Figure \ref{fig:Coupling_parScale_LIGGGHTS} shows the underlying coupling algorithm.

\begin{figure}[h!]
   \centering
   \includegraphics[scale=0.23,keepaspectratio=true]{Picture/liggghts_parscale_kopplung.eps}
   \caption{Coupling between \texttt{LIGGGHTS}\textsuperscript{\textregistered} and \texttt{ParScale} for one timestep including required coupling parameters and additional coupling options}
   \label{fig:Coupling_parScale_LIGGGHTS}
\end{figure}

For every timestep $t_n$ \texttt{LIGGGHTS}\textsuperscript{\textregistered} resolves the particle movement in the simulation domain in advance. After the computation is finished the coupling is realized through the updated surface temperature ($T_{surface}$), a heat transfer coefficient $\alpha$ and eventually a conductive heat flux due to collision of particle with different shell temperatures ($\dot{q}_{cond}$). \texttt{ParScale} initializes from the last proceeded time step $t_{n-1}$ and calculates all internal property fields according to the changes. Additional coupling options are available. \texttt{LIGGGHTS}\textsuperscript{\textregistered} is able to push a particle unique environment temperature (Fluid temperature) to \texttt{ParScale}. That enables \texttt{ParScale} to react due to changes in the environment temperature e.g. if the particle enters a region if a fixed temperature. Furthermore, the coupling provides the option to reset the value of the target property to a certain value. Due to the automatic sub-time stepping of \texttt{CVODE}, the internal \texttt{ParScale} timestep does not necessarily has to correspond to the DEM timestep from \texttt{LIGGGHTS}\textsuperscript{\textregistered}. If \texttt{ParScale} completed the simulation the surface temperatures of all particles are updated and taken into account by \texttt{LIGGGHTS}\textsuperscript{\textregistered} in timestep $t_{n+1}$. Furthermore, source terms due to reactions, core and volume averaged temperatures and fluxes are stored.\\
In section \ref{sec:verification_case} two verification cases, a transient cooling of a sphere and a heterogeneous reaction will be shown.

\section{Verification cases}
\label{sec:verification_case}

\subsection{Transient cooled sphere}
\label{sec:transient_cooling}
In the first  case a spherical particle is cooled down from a initial uniform temperature $T_0$. Table \ref{tab:properties_cooling_sphere} sums up the properties of the transient cooled sphere. Figure \ref{fig:Cooling_sphere} shows the comparison between the numerical solution by \texttt{ParScale} to the analytical solution provided by \cite{Crank1975} over a wide range of times. 

\begin{table}[h]
  \centering 
  \caption{Properties of the transient cooled sphere.}
   \begin{tabular}{llr}
      \hline 
        $c_p$                   &300       & [J$\,\text{m}^{-3}\,$K]\\   
       	$\rho$					&1000	   & [kg$\, \text{m}^{-3}$]\\
        $\lambda_s$				&1		   & [W$\, \text{m}^{-1} \,\text{K}^{-1}$]\\
        $\alpha_p$ 				&100	   & [W$\, \text{m}^{-2} \,\text{K}^{-1}$]\\
        $r_p$					&$5\cdot 10^{-3}$ & [m]\\
        $ T_0$					& 800& [K] \\
        $T_{enviro}$			& 300& [K]\\
        $t_1,t_2,t_3,t_4$ 		& 2,5,8,10 & [sec]\\
      \hline      
       \end{tabular}
   \setlength{\belowcaptionskip}{12pt}
   \label{tab:properties_cooling_sphere}
\end{table}


\begin{figure}[h!]
   \centering
   \includegraphics[scale=0.5,keepaspectratio=true]{Picture/transient_cooling_sphere.eps}
   \caption{Numerical and analytical solution for transient cooled sphere in comparison over different times. $t_1$ = 2 sec, $t_2$ = 5 sec, $t_3$ = 8 sec, $t_4$ = 10 sec.}
   \label{fig:Cooling_sphere}
\end{figure}

A very good agreement between analytical and numerical solution can be found for various times. 


\subsection{Heterogeneuous reation}
\label{sec:heterogeneuous_reaction}

This verification case validates the implementation of a single reaction according \cite{Wen} and offers a comparison to the presented analytical solution for Thiele Modulus of $\approx 3.16$. All parameters are chosen according to the copper oxidation based on \cite{Noorman2011}. The three basic assumpions are (i) isothermal particle temperature, (ii) first order gas reaction, (iii) reaction: $2 \, CO + O_2 \rightarrow \, 2\, CuO$. Figure \ref{fig:reaction_stages} show the comparism for both solid and fluid concentration inside the particle for the first and second stage reaction.

\begin{figure}[h!]
\centering
\subfloat[]{\label{main:a}\includegraphics[scale=.38]{Picture/concProfile_Stage1.eps}}
\hspace{0.3cm}
\subfloat[]{\label{main:b}\includegraphics[scale=.38]{Picture/concProfile_Stage2.eps}}\\
\caption{Comparism of solid and fluid concentrations for stage 1 (a) and stage 2 (b) of a heterogeneuous reation.}
\label{fig:reaction_stages}
\end{figure}

Figure \ref{fig:reaction_stages} (a) shows a very good agreement for the first stage reaction. Arising differences in the second stage (Figure \ref{fig:reaction_stages} (b)) are due to the pseudo-steady state modelling approach for the gaseous species in \cite{Noorman2011}. Since the differences are slightly higher than for the first stage reaction, Figure \ref{fig:Conversion} shows the overall Conversion. The comparism of analytical and numerical solution shows good agreement and the mean error is below 3 $\%$. 

\begin{figure}[h!]
   \centering
   \includegraphics[scale=0.5,keepaspectratio=true]{Picture/conversion.eps}
   \caption{Conversion charasteristics during an oxidation cycle.}
   \label{fig:Conversion}
\end{figure}


\newpage
\section{Simple shear flow}
\label{sec:simple_shear}

We now investigate the influcene of \texttt{ParScale} under well-controlled flow conditions. Therefore, particles are placed in a cubic periodic box (H/$\text{d}_\text{p}$ = 15) at various particle volume fractions $\phi_p \, =  \, (0.3 - 0.64)$. Lees–Edwards boundary conditions are applied to drive a homogeneous shear flow, with the shear gradient pointing in the y-direction. Along with the shear gradient, a temperature gradient in y-direction is applied. We determine the combination of Biot (Eqn. \ref{eqn:bi_number}) and Peclet (Eqn. \ref{eqn:Pe_number}) number as the two main non-dimensional infulence parameters as pointed out in \cite{Mohan2014} and \cite{Chialvo2012}. With the usage of \texttt{ParScale} at low Biot numbers, we expect agreeing results with \cite{Mohan2014}, \cite{Chialvo2012} and \cite{Vargas2001}. With higher Biot numbers the convecional flux will become dominant and the influence on the heat transfer rate important. Therefore intra particle profiles should be considered.

\begin{align}
Bi_t = \dfrac{\alpha\, d_p}{\lambda_p}
\label{eqn:bi_number}
\end{align}
where $\alpha$ is the heat transfer coefficient.
\begin{align}
Pe = \dfrac{\left(\dfrac{d_p}{2}\right)^2}{\dfrac{\lambda_p}{\rho_c \, c_p}} \, \cdot \dot{\gamma}
\label{eqn:Pe_number}
\end{align}  
where $\dot{\gamma}$ is the dimensional shear rate. The conductional reference flux is expressed as \cite{Chialvo2012}
\begin{align}
q_{cond,ref} = - \lambda_p \, \dfrac{\partial T}{\partial y_{middle}}.
\end{align}
where $q_{cond,ref}$ is the reference conductional heat flux and $y_{middle}$ is the size of the middle region. Table \ref{tab:properties_sheard_bed} shows the main non-dimensional parameters of the sheared bed simulation. For the disrect element method the open-source DEM solver \texttt{LIGGGHTS}\textsuperscript{\textregistered} is used. All other parameters, e.g. poisson ratio, coefficient of restitution and time step are in agreement with earlier simulations and can be found \cite{Mohan2014}.

\begin{table}[h]
  \centering 
  \caption{Dimensionless properties of the sheared bed particle case.}
   \begin{tabular}{llr}
      \hline 
        $T_{max}$				& 1 & \\
        $T_{min}$				& 0 & \\
        $d_p$					& 1 & \\
        $\phi_p$				& 0.3...0.64 &\\
        $Bi_t$					& $10^{-6}$...10 & \\
        $Pe$ 					& $0.25$& \\
      \hline      
       \end{tabular}
   \setlength{\belowcaptionskip}{12pt}
   \label{tab:properties_sheard_bed}
\end{table}


\begin{figure}[h!]
   \centering
   \includegraphics[scale=0.6,keepaspectratio=true]{Picture/sheared_bed.eps}
   \caption{Effect of the Biot number on the overall conductional heat flux for various particle volume fractions.}
   \label{fig:sheared_bed}
\end{figure}

As it can be seen in Figure \ref{fig:sheared_bed} the influence of the Biot number on the overall conductive flux needs to be condsidered. Even in low Biot number regimes ($Bi_t \approx 10^{-3}$) the conductive flux is underpredicted by up to 10 $\%$. Generally, even for higher particle volume fractions, at a Biot number of $Bi_t \approx 10^{-2}$ intra particle profiles should be taken into account for a sheared bed application. It is also shown that the influence of the Biot number on the conductive heat flux is dropping if the particle volume fraction is rising. 


\section{Conclusions}
\label{sec:conclusions}

We presented a novel open-source simulation tool \texttt{ParScale} which is published under LGPL licence and can be linked to any particle-based solver. We outlined the coupling to the open-source DEM solver \texttt{LIGGGHTS}\textsuperscript{\textregistered} and demonstrated the usage of \texttt{ParScale} with selected verification cases. A good agreement is found between numerical results produced by \texttt{ParScale} and analytical solutions available in literature. We extended a well-known sheared particle bed by taking intra particle temperature profiles into account. A result is that even in low Biot numbers regimes intra particle property profiles are of key importance and convective flux to the environment has to be considered. The current study was limited to selected particle volume fractions and Peclet numbers. Further work will feature wider ranges of these parameters. 

\section{Aknowledgement}
The authors acknowledge support by the European Commission through FP7 Grant agreement no. 604656 (“NanoSim”). \texttt{LIGGGHTS}\textsuperscript{\textregistered} is a registered trademark of DCS Computing GmbH.

\newpage
\bibliography{Literature20515}
%\end{thebibliography}

\end{document}


